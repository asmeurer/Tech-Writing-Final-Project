\newglossaryentry{cas}{
 % how the entry name should appear in the glossary
name={computer algebra system (CAS)},
 % how the description should appear in the glossary
 % I have to protect fragile commands
description={A computer program that performs mathematical operations symbolically},
 % how the entry should appear in the document text
text={CAS},
 % how the entry should appear the first time it is
 % used in the document text
first={computer algebra system (CAS)},
firstplural={computer algebra systems (CASs)}
}

\newglossaryentry{integration}{
name={integration},
description={One of two fundamental operations in \gls{calculus} (the
other being \gls{differentiation}).  Integration deals with the area under
curves.  Informally, $\int_a^b{f(x)\,dx}$ represents the area under the
curve defined by the function $f(x)$ from the points $x=a$ to $x=b$.  It
is an important result from \gls{calculus} called the Fundamental
Theorem of Gls{calculus} that integration and antidifferentiation, the
inverse operation of \gls{differentiation}, are essentially the same
thing}
}

\newglossaryentry{calculus}{
name={calculus},
description={The branch of mathematics that studies limits, derivatives,
integrals, and infinite series.  Informally, gls{calculus} is the study of
change.  Gls{calculus} is one of the most applied branch of mathematics; it
is used in every single branch of the physical sciences}
}

\newglossaryentry{differentiation}{
name={differentiation},
description={One of two fundamental operations in \gls{calculus} (the
other being integration).  \Gls{differentiation} deals with the change of
functions.  The function $\frac{df}{dx}$ is called the derivative of
$f$, and it informally represents the change, or slope, of the function
$f$ at any point $x$.  $\frac{df}{dx}$ is also sometimes denoted as
$f'(x)$, or simply $f'$ when the the independent variable $x$ is clear
or unspecified}
}

\newglossaryentry{elementary}{
name={elementary},
description={Roughly speaking, a function is gls{elementary} if it can be
represented as a combination of exponentials, logarithms, powers, and
trigonometric functions by addition, subtraction, multiplication,
division, and composition. For example, $\frac{\sin{(x^2 +
1)}}{\sqrt[3]{\ln{x}}}$ is gls{elementary}, but
$\frac{2}{\sqrt{\pi}}\int{e^{-x^2}\,dx}$, the \gls{error function}, is
not. See Bronstein's book~\cite{bronstein2005symbolic}, page 133, for a
more rigorous definition}
}

\newglossaryentry{error function}{
name={error function},
description={The function defined by \[\mathrm{erf}{(x)} =
\frac{2}{\sqrt{\pi}}\int{e^{-x^2}\,dx}.\]  This is the classic example
of an integral of an \gls{elementary} function that is not itself
\gls{elementary}.  The \gls{error function} is important in statistics.
In particular, it represents the cumulative distribution function of the
normal distribution (i.e., a bell curve), and its values are used to
calculate probabilities.  The fact that this function is nonelementary
implies that statistical computing packages must use numerical
techniques to calculate these values, as it does not have a closed form
solution}
}

\newglossaryentry{transcendental}{
name={transcendental},
description={A function is \gls{transcendental} if it is not
\gls{algebraic}.  A function is \textit{purely \gls{transcendental}} if
it does not contain any \gls{algebraic} components.  An important result
from analysis is that the functions $e^x$, $\ln{x}$, $\sin{x}$,
$\cos{x}$, and $\tan{x}$ are all \gls{transcendental}.  Roughly
speaking, a function is \gls{transcendental} if it contains one of
these, and it is purely \gls{transcendental} if it does not contain any
radicals.
%\footnote{There are exceptions to this rule.  For example,
%$e^{\frac{1}{2}\ln{x}}$ is not transcendental because it is equivalent to
%$\sqrt{x}$.  An important part of the \gls{transcendental} part of the
%Risch Algorithm involves making sure that the integrand really is
%\gls{transcendental}}  
For example, $e^{x + 1}$ is purely
\gls{transcendental}, $\sqrt[3]{\ln{x}}$ is \gls{transcendental} but not
purely \gls{transcendental}, and $\sqrt{x}$ is neither
\gls{transcendental} nor purely \gls{transcendental} (it is
\gls{algebraic})}
}

\newglossaryentry{algebraic}{
name={algebraic},
description={A function is \gls{algebraic} if it is the root of a
polynomial with coefficients that are \glspl{rational function} with
rational number coefficients.  For example, the function $\sqrt{x + 1}$
is \gls{algebraic} because it is the root of the polynomial $y^2 = x +
1$. A function that is not \gls{algebraic} is called
\gls{transcendental}}
}

\newglossaryentry{integrand}{
name={integrand},
description={The argument of an integral.  For example, the
\gls{integrand} of $\int{\sin{x}\,dx}$ is $\sin{x}$}
}

\newglossaryentry{rational function}{
name={rational function},
description={Any function that can be written as polynomial divided by a
polynomial.  For example, $\frac{x^2 + 1}{x^3 - x}$ is a \gls{rational
function}}
}
