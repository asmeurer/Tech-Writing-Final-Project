The first thing that I would like to do is to finish the
\gls{transcendental} part of the algorithm, so that given any
\gls{elementary} \gls{transcendental} function, \rischintegrate{} can
either produce its antiderivative or prove that no \gls{elementary}
antiderivative exists.  Right now, there are some corner cases in the
exponential and logarithmic algorithms that still need to be completed. 
Also, the trigonometric case has not been implemented yet.  Fortunately,
as was noted in Section \ref{description}, it is similar to the
exponential case that is already implemented.  After this, I will need
to test the algorithm very throughly to eliminate any bugs that might be
in it.

After this, there are many further avenues for \gls{integration} in
SymPy.  The most obvious next step would be to implement the algebraic
and mixed parts of the Risch Algorithm.  These are much harder to
implement, and they require more mathematical knowledge than I currently
posses, so this would have to come after I have taken some more classes
in algebraic curve theory, probably at a graduate school (there are no
courses on this subject at New Mexico Tech).  Unlike with Bronstein's
book, there are no definitive sources for the non-\gls{transcendental}
parts of the algorithm---Bronstein unfortunately died before being able
to write further volumes.  My references would have to come
from Trager's PhD thesis~\cite{trager1984integration} and Davenport's
book~\cite{davenport1984integration} on the algebraic part of the
algorithm, and from the journal articles.  Kauers~\cite{kauers2008integration} also suggests a heuristic method for
\gls{algebraic} \gls{integration} that could act as a stand in for the
otherwise complicated full algorithm.

Aside from the main Risch Algorithm, which deals with integrating
\gls{elementary} functions, there have been extensions to integrating
nonelementary functions, both for \gls{elementary} functions that do not
have \gls{elementary} antiderivatives and for integrals of nonelementary
functions.  The journal literature is full of methods for extending the
Risch Algorithm to these functions, for example, Adamchik's paper~\cite{adamchik1990hypergeometric} on integrating hypergeometric type
functions.

Another avenue would be to look at definite \gls{integration}.  The
Risch Algorithm deals with indefinite \gls{integration}, which is
equivalent to antidifferentiation (see the Glossary entry for
\gls{integration}).  Definite \gls{integration} deals with looking at
the area under the curve.  The Fundamental Theorem of Calculus gives a
simple formula to calculate a definite integral given an indefinite
integral.  However, some functions that do not have \gls{elementary}
indefinite integrals do have closed-form definite integrals over some
intervals.  A common example is \gls{error function} when evaluated from
$-\infty$ to $\infty$, since
$\frac{2}{\sqrt{\pi}}\int_{-\infty}^\infty{e^{-x^2}\,dx}=1$.

There does exist an algorithm to compute definite integrals.  It is not
a decision procedure like the Risch Algorithm, because it relies on the
ability to convert what is known as a Meijer G-Function into a
closed-form.  Roach's work on the method~\cite{roach1997meijerg} would
be a primary resource.  