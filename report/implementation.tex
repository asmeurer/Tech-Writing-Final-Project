Over the summer of 2010, as part of the Google Summer of Code
program\footnote{ See \url{http://code.google.com/soc/} for more
information about the program.}, I implemented the algorithms described
in Bronstein's book~\cite{bronstein2005symbolic} in the SymPy \gls{cas}.
As was noted in Section \ref{description}, the algorithm is very
complex. Bronstein's book only describes the \gls{transcendental} part
of the algorithm. I was able to finish the implementation of most of
this, so that the algorithm in SymPy can integrate most
\gls{transcendental} functions with exponentials and logarithms, or
prove that no such antiderivative exists.  I did not complete some
corner cases, and I also did not have time to write the trigonometric
case.

The Risch Algorithm is essentially a collection of polynomial
manipulation algorithms.  Fortunately, SymPy has an excellent polynomial
manipulation module, which was written by Mateusz Paprocki, who was also
my mentor for the Google Summer of Code program.  Therefore, I was able
to implement the algorithm mostly without having to deal with details on
lower levels, though I did have to learn how to use the polys module, as
it is called, pretty well, and there were a few cases where I had to fix
a few bugs in it.

Before my project, SymPy had a function called \texttt{integrate()} that
performed symbolic integration.  However, this function used a
combination of heuristics, similar to those described in Section
\ref{history}, and a simpler to implement heuristic version of the Risch
Algorithm called the Risch-Norman Algorithm, or the Parallel Risch
Algorithm.  \texttt{integrate()} was able to handle a wide class of
functions, but it was slow, and often failed to produce a result when
one existed.  It also had a tendency to hang on hard integrals, forcing
the user to interrupt the execution of the function.  Also, none of
these heuristics have the ability to prove that an \gls{elementary}
integral exists, they simply return the integral unevaluated, which is
the same thing that it does when it fails for an integral that does have
an closed form solution, so that there is no way to know if it failed
because an answer doesn't exist or because the heuristic could not find
one.

Because my implementation of the Risch Algorithm is not complete, and
because I wanted to be able to compare it to the old implementation, I
did not replace the code in \texttt{integrate()}.  Rather, I created a
new function, \rischintegrate{}, which acts as a front end to the Risch
Algorithm implementation.  The result was very succesful.  Not only
could the algorithm prove that a nonelementary integral doesn't exist
when that was the case, but it was also much faster, and could handle a
much larger set of functions (all \gls{transcendental} \gls{elementary}
functions, instead of just functions that are simple enough to pass
through the heuristic).

\subsection{Sample Session} 
\label{sample}

Below is an annotated sample session demonstrating \rischintegrate{} in action.  A few notes:
\begin{itemize}
\item SymPy runs in a Python intrepeter from the command line.
iSymPy is a script that automatically imports SymPy functions and
creates common variable names.  The below is running in \texttt{isympy
-o lex} in the iPython Python intrepeter.

\item SymPy uses unicode characters to pretty-print the equations. 
What is shown below is exactly what you would see in a actual iSymPy
session.

\item \texttt{\%timeit} is an iPython command that executes the command
multiple times and reports on the running time of the function. I have
placed these in for a few of the integrals to demonstrate the speed of
the algorithm.  All timeings come from a 2008 MacBook Pro with a 2.5 GHz
Intel Core 2 Duo processor and 4 GB of RAM.  The session is running in
Python 2.7.1, with the latest version of my \texttt{integration3}
development branch as the time of this writing.  

\item In some cases, I have wrapped \rischintegrate{} around
\texttt{factor()} or \texttt{cancel()}.  This is to simplify the result
so that it can be more compact or so that it is in the same form as
given before.

\item As with most \glspl{cas} integrators, the arbitrary constant, or
constant of integration, is ommited.  

\item The syntax for \rischintegrate{} is \texttt{risch\_integrate(f, x)},
where \texttt{f} is the expression to be integrated and \texttt{x} is
the integration variable.

\item In a few examples, I have included \texttt{handle\_first='exp'} at
the end of the \rischintegrate{} function call.  This affects the order in
which the ``parts'' discussed at the end of Section \ref{description}
are collected.  In the cases where I have used it here, the result is
simpler, or returns faster.  I have not yet completely figured out the
best way to do this automatically so that it produces the best result
without user intervention.  

\item Remember that \texttt{integrate()} is the old \gls{integration}
engine and \rischintegrate{} is the new engine.  I have run them both in a
few examples to compare results or timeings.  

\item If \rischintegrate{} returns an unevaluated integral, this means
that it has proven that the integral is not \gls{elementary}.  If it
fails (for example, because the case has not yet been implemented, or
because the equation is not \gls{elementary} purely
\gls{transcendental}), it will raise the \texttt{NotImplementedError}
exception.  None of the examples below raise this exception.

\item As with many \glspl{cas}, SymPy uses $\log$ instead of $\ln$. 
In SymPy, 1$\log$ means the natural logarithm, or log base $e$.
\end{itemize}

First, let us revisit Equation \ref{hard-integral}.  As we can see,
\rischintegrate{} has no problem generating the solution from Equation
\ref{hard-integral-sol}, and does so quickly.  The old
\texttt{integrate()} function hangs on this integral.

\begin{flushleft}
\includegraphics[width=1\textwidth]{sample1.pdf}\\
\includegraphics[width=1\textwidth]{sample2.pdf}\\
\includegraphics[width=1\textwidth]{sample3.pdf}\\
\includegraphics[width=1\textwidth]{sample4.pdf}
\end{flushleft}

Next, let us look at those equations from Appendix
\ref{nonelementary_examples_appendix} that do not have
trigonmetric parts.  \rischintegrate{} returns an unevaluated integral for
each, meaning that it can prove that each is nonelementary.  Due to a
current limitation in the algorithm, we must first rewrite Equations
\ref{x_to_the_x} and \ref{x_to_the_neg_x} as $e^{x\ln{x}}$ and
$e^{-x\ln{x}}$, respectively.  These forms are mathematically equivalent.

\begin{flushleft}
\includegraphics[width=1\textwidth]{sample5.pdf}
\includegraphics[width=1\textwidth]{sample6.pdf}
\includegraphics[width=1\textwidth]{sample7.pdf}
\includegraphics[width=1\textwidth]{sample8.pdf}
\includegraphics[width=1\textwidth]{sample9.pdf}
\includegraphics[width=1\textwidth]{sample10.pdf}
\end{flushleft}

Here is an example of the sort of thing discussed in Section
\ref{history}.  The old \texttt{integrate()} cannot handle this
integral, so it returns an unevaluated integral\footnote{Unlike with
\rischintegrate{}, when \textttt{integrate()} returns an unevalauted
integral, it does not necessarily mean that the integral is
nonelementary.}.  \rischintegrate{} is able to handle the integral just
fine.

\begin{flushleft}
\includegraphics[width=1\textwidth]{sample11.pdf}
\includegraphics[width=1\textwidth]{sample12.pdf}
\end{flushleft}
