Over the summer of 2010, as part of the Google Summer of Code
program\footnote{ See \url{http://code.google.com/soc/} for more
information about the program.}, I implemented the algorithms described
in Bronstein's book~\cite{bronstein2005symbolic} in the SymPy \gls{cas}.
As was noted in Section \ref{description}, the algorithm is very
complex. Bronstein's book only describes the \gls{transcendental} part
of the algorithm. I was able to finish the implementation of most of
this, so that the algorithm in SymPy can integrate most
\gls{transcendental} functions with exponentials and logarithms, or
prove that no such antiderivative exists.

The Risch Algorithm is essentially a collection of polynomial
manipulation algorithms.  Fortunately, SymPy has an excellent polynomial
manipulation module, which was written by Mateusz Paprocki, who was also
my mentor for the Google Summer of Code program.  Therefore, I was able
to implement the algorithm mostly without having to deal with details on
lower levels, though I did have to learn how to use the polys module, as
it is called, pretty well, and there were a few cases where I had to fix
a few bugs in it.

\emph{This will, among other things, include some
sample sessions demonstrating the algorithm.}
