Over the summer of 2010, as part of the Google Summer of Code
program\footnote{ See \url{http://code.google.com/soc/} for more
information about the program.}, I implemented the algorithms described
in Bronstein's book~\cite{bronstein2005symbolic} in the SymPy \gls{cas}.
As was noted in Section \ref{description}, the algorithm is very
complex. Bronstein's book only describes the \gls{transcendental} part
of the algorithm. I was able to finish the implementation of most of
this, so that the algorithm in SymPy can integrate most
\gls{transcendental} functions with exponentials and logarithms, or
prove that no such antiderivative exists.  I did not complete some
corner cases, and I also did not have time to write the trigonometric
case.

The Risch Algorithm is essentially a collection of polynomial
manipulation algorithms.  Fortunately, SymPy has an excellent polynomial
manipulation module, which was written by Mateusz Paprocki, who was also
my mentor for the Google Summer of Code program.  Therefore, I was able
to implement the algorithm mostly without having to deal with details on
lower levels, though I did have to learn how to use the polys module, as
it is called, pretty well, and there were a few cases where I had to fix
a few bugs in it.

Before my project, SymPy had a function called \texttt{integrate()} that
performed symbolic integration.  However, this function used a
combination of heuristics, similar to those described in Section
\ref{history}, and a simpler to implement heuristic version of the Risch
Algorithm called the Risch-Norman Algorithm, or the Parallel Risch
Algorithm.  \texttt{integrate()} was able to handle a wide class of
functions, but it was slow, and often failed to produce a result when
one existed.  Also, none of these heuristics have the ability to prove
that an \gls{elementary} integral exists, they simply return the
integral unevaluated, which is the same thing that it does when it fails
for an integral that does have an closed form solution, so that there is
no way to know if it failed because an answer doesn't exist or because
the heuristic could not find one.

Because my implementation of the Risch Algorithm is not complete, and
because I wanted to be able to compare it to the old implementation, I
did not replace the code in \texttt{integrate()}.  Rather, I created a
new function, \rischintegrate{}, which acts as a front end to the Risch
Algorithm implementation.  The result was very succesful.  Not only
could the algorithm prove that a nonelementary integral doesn't exist
when that was the case, but it was also much faster, and could handle a
much larger set of functions (all \gls{transcendental} \gls{elementary}
functions, instead of just functions that are simple enough to pass
through the heuristic).

\subsection{Sample Session} 
\label{sample} 
\includegraphics[width=1\textwidth]{sample1.pdf}