Bronstein's ``Symbolic Integration Tutorial''
\cite{bronstein1998symbolic} gives an overview of the entire Risch
Algorithm.  The algorithm has three parts:  the \gls{transcendental}
part, the \gls{algebraic} part, and the mixed part.  The focus of this
paper is on the \gls{transcendental} part, so we only look at that.

The integration algorithm works recursively.  It integrates whatever the
highest level function is first, and works its way down.  For example,
the highest level function in $e^{\sin{(\ln{x})}}$ is
$e^{\sin{(\ln{x})}}$, followed by $\sin{(\ln{x})}$, $\ln{x}$, and
finally $x$, which is always the lowest level. The first three parts of
the algorithm take an \gls{integrand} and produce part of the integral
and another integral, which is in some sense ``simpler''. These parts
are called the Hermite Reduction, the Polynomial Reduction, and the
Rothstein-Trager Residue Reduction, respectively.  The Hermite Reduction
can always be applied to any function, though sometimes the part of the
solution it returns is just 0 and the reduced integral is the same as
the original integral.  The Polynomial Reduction can only be applied for
a certain class of functions (trigonometric functions fall into this
class), though when it can be applied, it too always returns some
reduced integral, which may be the same as the original.  The Residue
Reduction, which finds the logarithmic part of the integral, or the
$c_i$ and $u_i$ from Equation \ref{liouville's theorem} is the first
part of the algorithm that can potentially prove that no elementary
integral exists.  In this case, the reduction has shown that no
logarithmic part can exist, and hence, by Liouville's Theorem, no
\gls{elementary} integral can exist.  Not all non-\gls{elementary}
integrals are proven so at this stage, but, for example, the integral in
Equation \ref{li} in Appendix \ref{nonelementary_examples_appendix} is
found to be non-\gls{elementary} at this stage.

After this point, in the algorithm, what remains is (in some sense) a
polynomial in the highest level function.  The algorithm here splits
into three parts, depending on what the highest level function.  If it
is a logarithm, it goes to the Primitive Case; if it is an exponential,
it goes to the Exponential Case, and if it is a tangent, it goes to the
Tangent Case\footnote{One disadvantage about the integration algorithm
is that we must first write all sines and cosines in terms of tangents,
generally using the formulas $\sin{x} = \frac{2\tan{(\frac{x}{2}})}{1 +
\tan^2{(\frac{x}{2})}}$ and $\cos{x} = \frac{1 -
\tan^2{(\frac{x}{2})}}{1 + \tan^2{(\frac{x}{2})}}$.  After integrating,
we will still have $\tan{(\frac{x}{2})}$ terms, so if we want sines and
cosines again, we must then apply the formula $\tan{(\frac{x}{2})} =
\frac{\sin{x}}{1 + \cos{x}}$ and simplify.}.  Each of these cases
reduces the integral to an equivalent differential equation.  The
solution to this differential can immediately be converted to the
solution of the integral, or, if it can be shown that the differential
equation has no solution, then it will also have been shown that the
integral is not \gls{elementary}.  For example, for the Exponential
Case, the problem of computing
\begin{equation}
\label{exponential case}
\int{p(x)e^{q(x)}\,dx}
\end{equation}
is equivalent to finding a solution $y$ to the equation $y' + fy = g$,
where $f$ and $g$ are derived from $p(x)$ and $q(x)$.\footnote{Note that
this is not the same as the form given in Equation \ref{SIN exponential
form} because here $p(x)$ and $q(x)$ are not limited to be
\glspl{rational function}.  In fact, that heuristic applies only a very
simple part of the Exponential Case algorithm.}  The equation $y' + fy =
g$ is called the Risch Differential Equation, and much of Risch's work
on completing the integration algorithm involved completing a complete
algorithm for solving it.  The equations for the Primitive Case and the
Tangent Case are called the Parametric Risch Differential Equation and
the Coupled Risch Differential Equation System, respectively.  Each of
these is very similar to the Risch Differential Equation, and the
algorithms for solving them are similar.

Once the specific algorithm for solving the differential equation has
been applied, either it will have shown that the differential equation
does not have a solution, in which case it will have proven that the
remaining integral is not elementary, or it will return part of the
solution and an integral that only contains functions of the lower
levels.  The entire algorithm is applied recursively, until either some
part proves that the integral is non-\gls{elementary}, or the lowest
level, $x$, is reached, after which the integration will be complete.  