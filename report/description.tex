Bronstein's ``Symbolic Integration Tutorial''
\cite{bronstein1998symbolic} gives an overview of the entire Risch
Algorithm.  The algorithm has three parts:  the \gls{transcendental}
part, the \gls{algebraic} part, and the mixed part.  The focus of this
paper is on the \gls{transcendental} part, so we will only look at that.

The \gls{integration} algorithm works recursively.  It integrates whatever the
highest ``level'' function is first, and works its way down.  For example,
the highest level function in $e^{\sin{(\ln{x})}}$ is
$e^{\sin{(\ln{x})}}$, followed by $\sin{(\ln{x})}$, $\ln{x}$, and
finally $x$, which is always the lowest level. The first three parts of
the algorithm take an \gls{integrand} and produce part of the integral
and another integral, which is in some sense ``simpler''. These parts
are called the Hermite Reduction, the Polynomial Reduction, and the
Rothstein-Trager Residue Reduction, respectively.  The Hermite Reduction
can always be applied to any function, though sometimes the part of the
solution it returns is just 0 and the reduced integral is the same as
the original integral.  The Polynomial Reduction can only be applied for
a certain class of functions (trigonometric functions fall into this
class), though when it can be applied, it too always returns some
reduced integral, which may be the same as the original.  The Residue
Reduction, which finds the logarithmic part of the integral (the
$c_i$ and $u_i$ from Equation \ref{liouville's theorem}) is the first
part of the algorithm that can potentially prove that no \gls{elementary}
integral exists.  In this case, the reduction has shown that no
logarithmic part can exist, and hence, by Liouville's Theorem, no
\gls{elementary} integral can exist.  Not all nonelementary
integrals are proven so at this stage, but, for example, the integral in
Equation \ref{li} in Appendix \ref{nonelementary_examples_appendix} is
found to be nonelementary at this stage.

After this point, in the algorithm, what remains is (in some sense) a
polynomial in the highest level function.  The algorithm here splits
into three parts, depending on what type the highest level function is.  If it
is a logarithm, it goes to the primitive case; if it is an exponential,
it goes to the exponential case, and if it is a tangent, it goes to the
tangent case\footnote{One disadvantage about the \gls{integration} algorithm
is that we must first write all sines and cosines in terms of tangents,
generally using the formulas $\sin{x} = \frac{2\tan{(\frac{x}{2}})}{1 +
\tan^2{(\frac{x}{2})}}$ and $\cos{x} = \frac{1 -
\tan^2{(\frac{x}{2})}}{1 + \tan^2{(\frac{x}{2})}}$.  After integrating,
we will still have $\tan{(\frac{x}{2})}$ terms, so if we want sines and
cosines again, we must then apply the formula $\tan{(\frac{x}{2})} =
\frac{\sin{x}}{1 + \cos{x}}$ and simplify.}.  Each of these cases
reduces the integral to an equivalent differential equation.  The
solution to this differential equation can immediately be converted to the
solution of the integral, or, if it can be shown that the differential
equation has no solution, then it will also have been shown that the
integral is not \gls{elementary}.  For example, for the exponential
case, the problem of computing
\begin{equation}
\label{exponential case}
\int{p(x)e^{q(x)}\,dx}
\end{equation}
is equivalent to finding a solution $y$ to the equation $y' + fy = g$,
where $f$ and $g$ are derived from $p(x)$ and $q(x)$.\footnote{Note that
this is not the same as the form given in Equation \ref{SIN exponential
form} because here $p(x)$ and $q(x)$ are not limited to be
\glspl{rational function}.  In fact, that heuristic implements only a
very simple part of the Risch Differential Equation algorithm.}  The
equation $y' + fy = g$ is called the Risch Differential Equation, and
much of Risch's work on completing the \gls{integration} algorithm
involved completing a complete algorithm for solving it.  The equations
for the Primitive Case and the Tangent Case are called the Parametric
Risch Differential Equation and the Coupled Risch Differential Equation
System, respectively.  Each of these is very similar to the Risch
Differential Equation, and the algorithms for solving them are similar.

Once the specific algorithm for solving the differential equation has
been applied, either it will have shown that the differential equation
does not have a solution, in which case it will have proven that the
remaining integral is not \gls{elementary}, or it will return part of
the solution and an integral that only contains functions of the lower
levels.  The entire algorithm is applied recursively on this integral,
until either some part proves that the integral is nonelementary, or the
lowest level, $x$, is reached, after which the \gls{integration} will be
complete.

Among all these parts, the entire algorithm is very complex.  For
example, the algorithm for solving the Risch Differential Equation has
five parts, each of which must be split up into separate cases depending
on what the type of next lower level of the integral is. Also, before
any \gls{integration} can take place, the \gls{integrand} must be parsed
into these ``levels''.  This process is complicated by the fact that it
must be verified that the integral is purely \gls{transcendental} (see
the Glossary entry for \gls{transcendental}).  This is important because
a function may appear to be purely \gls{transcendental} but really be
\gls{algebraic} or \gls{algebraic} over some \gls{transcendental}
extension.  For example, the function $e^{\frac{1}{2}\ln{x}}$ appears to
be transcendental, but by the logarithmic power identity and the fact
that the exponential is the inverse of the logarithm, it is actually
identically equal to $\sqrt{x}$.  There is an algorithm, called the
Structure Theorems, for doing this.  This must be implemented or else
any result that an integral is nonelementary might not be valid if the
\gls{integrand} is really \gls{algebraic}.  

The algorithm can take any elementary function and either prove that the
function does not have an \gls{elementary} antiderivative, or else
produce one if it does.  For this reason, the Risch Algorithm is
sometimes called a decision procedure, because it can completely decide
the problem of symbolic \gls{integration} for elementary functions.  
