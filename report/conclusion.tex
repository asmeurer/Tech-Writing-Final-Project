The problem of symbolically integrating \gls{elementary} functions has
been completely solved since the 1970's in the form of the Risch
Algorithm, but due to its complexity few \glspl{cas} other than the
large ones (such as Maple and Mathematica) implement it.  Of the open
source \glspl{cas}, Axiom is the only one aside from SymPy that
implements the algorithm.  Therefore, it is of great benefit to have an
implementation of the algorithm.  The full algorithm is a great
improvement over the heuristic methods that most \glspl{cas} apply. 
First, it can actually be faster.  The reason for this is that the Risch
Algorithm integrates a function directly, whereas heuristic methods
attempt to guess what the solution looks like or what type it is.  Also,
the \rischintegrate{} implementation in SymPy is faster because it uses
the faster polys module, whereas the older \texttt{integrate()} uses
older, slower parts of the system.

Second, the Risch Algorithm is a decision procedure.  This has two
benefits.  First, it means that the algorithm can always produce a
solution, no matter how complex the \gls{integrand} is.  Second, it
means that it can prove that no \gls{elementary} solution exists
whenever this is the case. This has both theoretical and practical uses.
 If an integral can be shown not to have a closed form solution, then it
will be more beneficial to use numerical techniques to calculate the
values of that function.  Also, it is very interesting from a
theoretical standpoint that not only can this problem be solved
completely, but it can be done so programmatically by a computer.

The implementation of the \gls{transcendental} part of the algorithm in
SymPy, though not yet entirely complete, has been very successful. 
And after it is completed, there are plenty of avenues to continue work on
symbolic \gls{integration} in SymPy.
