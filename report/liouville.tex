At the core of the \gls{integration} algorithm lies an important result
called Liouville's Theorem.  Roughly, the theorem states that if a
function $f$ has an \gls{elementary} integral, then the integral can
always be written in the form
\begin{equation}
\label{liouville's theorem}
\int{f} = v + \sum_{n=1}^m{c_i\ln{u_i}}
\end{equation}
where $v$ and the $u_i$ are ``parts'' of $f$, and the $c_i$ are
constants.  In other words, any \gls{elementary} integral does not add
any more \gls{elementary} parts from the original, except for a finite
number of logarithms of those parts.  Again, we use the term ``parts''
here in a very rough sense; the theorem is actually very rigorously
defined (see Bronstein's book, page 145~\cite{bronstein2005symbolic}). 
This theorem is important because it gives the exact form that any
\gls{elementary} antiderivative must have. Therefore, if we can prove that
there is no function $g$ of the form of Equation \ref{liouville's
theorem} such that $g'=f$, then we have proven that $f$ does not have an
\gls{elementary} antiderivative. 