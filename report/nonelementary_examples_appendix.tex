\renewcommand{\thefootnote}{\fnsymbol{footnote}}	
The following are some classic examples of integrals of transcendental
elementary functions that do not have closed form (\gls{elementary})
solutions.

\begin{equation}
\label{erf}
\int{e^{-x^2}\,dx};
\end{equation}
\begin{equation}
\label{exp_x_squared}
\int{e^{x^2}\,dx};
\end{equation}
\begin{equation}
\label{li}
\int{\frac{1}{\ln{x}}\,dx};
\end{equation}
\begin{equation}
\label{x_to_the_x}
\int{x^x\,dx};
\end{equation}
\begin{equation}
\label{x_to_the_neg_x}
\int{x^{-x}\,dx};
\end{equation}
\begin{equation}
\label{Si}
\int{\frac{\sin{x}}{x}\,dx};
\end{equation}
\begin{equation}
\label{Ci}
\int{\frac{\cos{x}}{x}\,dx};
\end{equation}
\begin{equation}
\label{exp_x_ln_x}
\int{e^x\ln{x}\,dx}
\end{equation}

\textbf{Notes}
\begin{itemize}
\item Integral \ref{erf} is, except for a constant factor, the
\gls{error function}.  See the Glossary.
\item Integral \ref{li} is is often denoted by $\mathrm{li}(x)$
(Logarithmic Integral).
\item The integrals \ref{Si} and \ref{Ci} are often denoted by
$\mathrm{Si}(x)$ (Sine Integral) and $\mathrm{Ci}(x)$ (Cosine Integral),
respectively.  Similar integrals with $e^x$, $\cosh{x}$, and $\sinh{x}$
are also nonelementary and have corresponding named special functions.
\item Integrals \ref{erf}, \ref{exp_x_squared}, \ref{li},
\ref{x_to_the_x}, \ref{x_to_the_neg_x}, and \ref{exp_x_ln_x}, as well as
the additional integrals noted in the previous bullet point, can all be
proved to be non\-el\-e\-men\-tary using \texttt{risch\_integrate()}.
The remaining integrals are trigonometric and hence not implemented yet
(see Section \ref{future}).  However, the fact that they are not
elementary is can be shown by applying the relevant part of the Risch
Algorithm to them. For example, the steps  for integral \ref{Si} are
worked out in in Bronstein's book \cite{bronstein2005symbolic}, page 169.
\item There are many classic nonelementary integrals of
non-transcendental elementary functions, such as the hyperbolic
integrals.  We do not look at these here, as the focus of this paper is
chiefly on the transcendental part of the Risch Algorithm.
\end{itemize}