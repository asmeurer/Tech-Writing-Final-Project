\renewcommand{\thefootnote}{\fnsymbol{footnote}}	
The following are some classic examples of integrals of elementary
functions that do not have closed form (\gls{elementary}) solutions.

\begin{equation}
\label{erf}
\int{e^{-x^2}\,dx};
\end{equation}
\begin{equation}
\label{exp_x_squared}
\int{e^{x^2}\,dx};
\end{equation}
\begin{equation}
\label{li}
\int{\frac{1}{\ln{x}}\,dx};
\end{equation}
\begin{equation}
\label{exp_x_squared}
\int{e^{x^2}\,dx};
\end{equation}
\begin{equation}
\label{x_to_the_x}
\int{x^x\,dx};
\end{equation}
\begin{equation}
\label{x_to_the_neg_x}
\int{x^{-x}\,dx};
\end{equation}
\begin{equation}
\label{x_to_the_x}
\int{x^x\,dx};
\end{equation}
\begin{equation}
\label{Si}
\int{\frac{\sin{x}}{x}\,dx};
\end{equation}
\begin{equation}
\label{Ci}
\int{\frac{\cos{x}}{x}\,dx};
\end{equation}

\textbf{Notes}
\begin{itemize}
\item Equation \ref{erf} is, except for a constant factor, the
\gls{error function}.  See the Glossary.
\item Equation \ref{li} is is often denoted by $\mathrm{li}(x)$.
\item Equations \ref{Si} and \ref{Ci} are often denoted by
$\mathrm{Si}(x)$ and $\mathrm{Ci}(x)$, respectively.  Many similar
integrals also are named special functions.
\end{itemize}