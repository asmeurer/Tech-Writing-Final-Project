\Gls{calculus} is perhaps the most important branch of mathematics with
regard to the physical sciences.  \Gls{calculus} has two important
operations, \gls{differentiation} and its inverse operation,
antidifferentiation, or \gls{integration}.  \Gls{differentiation} is
easy to apply to any \gls{elementary} function simply by repeatedly
applying some simple rules, such as the sum, product, and chain rules.
Furthermore, it is not hard to see from the definitions that the
derivative of an \gls{elementary} function is again \gls{elementary}.

\Gls{integration} is a different story.  There are many heuristics that
can be applied to simple integrals---\gls{integration} by substitution,
\gls{integration} by parts, trigonometric \gls{integration}, and
trigonometric substitution are a few---but it is easy to come up with
arbitrarily complex expressions for which none of these methods apply
but for which there exist an \gls{elementary} solution simply by taking
differentiating any elementary function of arbitrary complexity.  For
example, by differentiating
\begin{equation}
\label{hard-integral-sol}
    \frac{\left(e^{x^{2} + 1}\right)^{2}}{x - \ln{x}}
\end{equation}
and simplifying we obtain
\begin{equation}
\label{hard-integral}
    \frac{\left(1 + e^{x^{2}}\right) \left(1 - x - 4 x^{2} e^{x^{2}}
    \ln{x} - x e^{x^{2}} + 4 x^{3} e^{x^{2}} +
    e^{x^{2}}\right)}{x \left(x -
    \ln{x}\right)^{2}},
\end{equation}
which is complicated enough that none of the above mentioned simple
methods can be applied to it.  It is also unlikely to be found in any
integration table.  Of course, we could create arbitrarily more complex
expressions that have elementary antiderivatives; for example, we
could further differentiate equation \ref{hard-integral}. 

But the problem is complicated further by the fact that there exist
\gls{elementary} functions for which there does not exist any
\gls{elementary} antiderivatives.  For example, it can be shown that
the \gls{error function} (see the Glossary) is not elementary.  Quite a
few very simple elementary functions do not have elementary
antiderivatives.  See Appendix \ref{nonelementary_examples_appendix}
for some common examples. Therefore, using a heuristic based
\gls{integration} method might fail altogether simply because an a
solution does not exist .

Fortunately, despite its complexity, it turns out that the problem of
integrating elementary functions or showing that such integral exists
can be solved algorithmically.  The algorithm is called Risch's
Algorithm, because it was first proved to be the case by Robert Risch in
1969~\cite{risch1969problem}.  The ``algorithm'' is very complex
collection of sub-algorithms.  I implemented the simplest of these, the
\gls{transcendental} part, which deals with integrating
\gls{transcendental} functions (see the Glossary), in the SymPy
\gls{cas} over the summer of 2010.
