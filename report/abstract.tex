\Gls{integration} is one of the two important operators from \gls{calculus},
the other being \gls{differentiation}.  Unlike \gls{differentiation},
however, symbolic \gls{integration} (i.e., indefinite \gls{integration})
of \gls{elementary} functions is not a straightforward process. The
methods taught in \gls{calculus}, such as \gls{integration} by substitution,
\gls{integration} by parts, trigonometric \gls{integration}, and
trigonometric substitution are only heuristics that can be applied to a
special class of \gls{elementary} functions. It turns out that there
exists an algorithm, called the Risch Algorithm, which not only gives a
complete algorithm for symbolic \gls{integration}, but it can also prove
that no \gls{elementary} antiderivative can exist for the integral
whenever this is the case. In this paper, we discuss the Risch
Algorithm, and the author's implementation of the \gls{transcendental}
part of that algorithm in the SymPy computer algebra system.  In
particular, we look at the history of the algorithm, give an overview of
the algorithm, look the implementation in SymPy, and consider some
possible future work in the area of symbolic \gls{integration} for
SymPy.
