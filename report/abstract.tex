Integration is one of the two important operators from calculus, the
other being differentiation.  Unlike differentiation, however, symbolic
integration (i.e., indefinite integration) of elementary functions is
not a straightforward process. The methods taught in calculus, such as
integration by substitution, integration by parts, trigonometric
integration, and trigonometric substitution are only heuristics that can
be applied to a special class of elementary functions. It turns out that
there exists an algorithm called the Risch Algorithm, which not only
gives a complete algorithm for symbolic integration but it can also
prove that no elementary antiderivative can exist for the integral
whenever that is the case. In this paper, we discuss the Risch
Algorithm, and the author's implementation of the transcendental part of
that algorithm in the SymPy computer algebra system.  In particular, we
look at the history of the algorithm, the implementation in SymPy, and
some possible future work in the area of symbolic integration for SymPy.
