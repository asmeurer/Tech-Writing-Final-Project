Before Risch discovered the \gls{integration} algorithm, the \glspl{cas}
that existed used heuristic methods to integrate common functions.
Moses, in his 1971 ``Stormy Decade'' paper~\cite{moses1971symbolic},
details the methods that his SIN \gls{cas} used for \gls{integration}. SIN was
written before the Risch Algorithm was discovered, so this gives an
insight to what the \gls{integration} algorithms in \glspl{cas} looked like
before that, when they only used heuristic methods.  The method used is
a three step method: first try to apply the simplest possible heuristic,
which basically involves trying to pattern match $\int{c\
\mathrm{op}(u(x))u'(x)\,dx}$. When an \gls{integrand} has this form, the
integral can be reduced to simply $c\int{\mathrm{op}(z)\,dz}$, by making
the substitution $z=u'(x)\,dx$.  If $\mathrm{op}$ has a known
antiderivative, then the entire integral can be computed.  This is
essentially the ``$u$-substitution'' heuristic method from calculus.  If
this method fails, which it will for all but the most simple integrals,
the second stage is applied.  At this stage, the \gls{integrand} is
matched against eleven more complicated but simple heuristics for common
integrals of exponentials, logarithms, trigonometric, and
\gls{algebraic} functions.  For example, some of the methods are
\gls{integration} of exponentials via the substitution $y=c^x$,
\gls{integration} of expressions with $\sqrt{ax^2 + bx + c}$ via an
arctrigonometric substitution, and the specific sub-algorithm of the
Risch Algorithm that deals with integrals of the form
\begin{equation}
\label{SIN exponential form}
\int{p(x)e^{q(x)}\,dx},
\end{equation}
where $p(x)$ and $q(x)$ are \glspl{rational function} in $x$ (this is
the only part of the SIN algorithm that can determine that an integral
is not \gls{elementary}). The third stage applies more advanced methods
such as partial fractions or further parts of the Risch Algorithm.  Each
stage of the procedure is easier to compute, but is less likely to
produce an answer than those that come after it.

The original algorithm by Risch, while a decision procedure, was only an
algorithm in the technical sense, as many parts were far too complicated
to ever be implemented in a \gls{cas}.  It wasn't until
Davenport~\cite{davenport1984integration},
Trager~\cite{trager1984integration},
Bronstein~\cite{bronstein2005symbolic}, and others came along that these
parts were simplified to the point that they could actually be
implemented.  In particular, the \gls{algebraic} part relied on some
complicated \gls{algebraic} curve theory, which was simplified to some degree
by Davenport and also Trager.  However, as the focus of this paper is
mainly on the \gls{transcendental} part of the algorithm, we will not
look much further into this.  