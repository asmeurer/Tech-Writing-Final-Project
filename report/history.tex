\emph{The relevant references here
are~\cite{moses1971symbolic,risch1969problem}}

Before Risch discovered the \gls{integration} algorithm, the \glspl{cas}
that existed used heuristic methods to integrate common functions. Moses
in his 1971 ``Stormy Decade'' paper \cite{moses1971symbolic} details the
methods that his SIN \gls{cas} used for integration. SIN was written
before the Risch algorithm was discovered, so this gives an insight to
what the integration algorithms in \glspl{cas} looked like before that,
when they only used heuristic methods.  The method used is a three step
method: first try to apply the simplest possible heuristic, which
basically involves trying to pattern match $\int{c\
\mathrm{op}(u(x))u'(x)\,dx}$.  When an \gls{integrand} has this form,
the integral can be reduced to simply $c\int{\mathrm{op}(z)\,dz}$, by
making the substitution $z=u'(x)\,dx$.  If $\mathrm{op}$ has a known
antiderivative, then the entire integral can be computed.  This is
essentially the ``$u$-substitution'' heuristic method from calculus.  If
this method fails, which it will for all but the most simple integrals,
the second stage is applied.  At this stage, the \gls{integrand} is
matched against eleven more complicated but simple heuristics for common
integrals of exponentials, logarithms, trigonometric, and algebraic
functions.  For example, some of the methods are integration of
exponentials via the substitution $y=c^x$, integration of expressions
with $\sqrt{ax^2 + bx + c}$ via an arc\-trig\-o\-no\-met\-ric
substitution, .
