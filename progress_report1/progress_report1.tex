\documentclass[12pt]{article}
\usepackage{amsmath}
\usepackage{amsfonts}
\usepackage[latin1]{inputenc}
\usepackage{amsmath}
\usepackage{amsfonts}
\usepackage{amssymb}
\usepackage{makeidx}
\usepackage{tabularx}
\usepackage{url}
%\usepackage[T1]{fontenc}
\newcommand{\BibTeX}{{\sc Bib}\TeX}
\newcommand{\bibtex}{{\sc Bib}\TeX\ }
\newcommand{\latex}{\LaTeX\ }
\begin{document}
\title{Progress Report 1 for Report on the Risch Algorithm for Symbolic
Integration and Implementation in the Sym\-Py Computer Algebra System}
\author{Aaron Meurer}
\date{November 12, 2010}
\maketitle

% Include the following information:
% --Secondary research conducted
% --Primary research conducted (if any)
% --Any changes or modifications in your report topic or outline
% --Any changes to your time schedule
% --Any questions, concerns, etc.
% --A short, annotated bibliography with 5 secondary sources (in an
% appendix). For each source, provide a citation using whatever format
% you intend to use for the final report. Also, for each source, include
% a brief paragraph summarizing the source and explaining how you intend
% to use it in your report.

\section{Research Conducted}
\subsection{Primary Research}
For all intents and purposes, the primary research for my project has
already been conducted.  As I said in the proposal, I spent last summer
learning and implementing the Transcendental Risch Algorithm for
Symbolic Integration. Because I was implementing the algorithm, I had to
fully understand it, both from a mathematical and a implementational
perspective.  Therefore, I have read quite closely almost the entirety
of Bronstein's book \cite{bronstein2005symbolic}.  In other words, this was
more than just a cursory reading.  In order to implement an algorithm,
you have to understand exactly what is happening mathematically, so that
you can fix problems when they arrise.  Also, sometimes the algorithms
given were ambiguous, requiring me to understand what should be
happening enough to implement it, because actual implementations in
programming languages cannot be ambiguous in any way, by their nature.

\subsection{Secondary Research}
Secondary research is actually the thing that I need to do more of. I
still need to read through most of the citations that I go through in
Appendix \ref{annotated-bib}.  I think that some of these may end up
changing for the final report, depending on exactly what direction I
want to take with it.  For example, the sources I have now mostly relate
to things that I would want to do with the algorithm in the future that
go beyond Bronstein's book.  For now, that is the plan, because I don't
see many other things that I could do with secondary sources.  But, for
example, one thing that I might choose to do instead would be to look at
what other people in the field have done.  That would require a whole
different secondary source set.  This would be more difficult, though,
as it would require me reading source code for other programs, rather
than reading articles from journals.

\section{Modifications}
Nothing has changed with respect to the outline, though some parts have
been refined, as per the suggestions at the conference.  The timeline
may need to be modified a little bit, as I have not written as much of
the paper itself at this point as I originally had planned.  However, I
do not see it as an issue, as I am already ahead with the actual
research part.

\section{Questions}
I have one question.  Should I cite the source code for the algorithm
that I implemented in the references?  The source code is freely
available (it is open source, under the BSD license) at
\url{http://github.com/asmeurer/sympy/tree/integration3}.  If so, I'm
not exactly sure how it will look.  Citing source code is not a standard
format in \BibTeX.  

Another issue is that Sym\-Py is a collaborative project, written by many
people, so the majority of the source code was actually written by
someone else.  You can see the changes that are actually by me at
\url{http://github.com/asmeurer/sympy/commits/integration3} (any change
by \texttt{asmeurer} is by me).

\appendix
\section{Annotated Bibliography}
\label{annotated-bib}
\begin{itemize}
\item These sources all have a common thread: they all are about things
that I could use to extend the algorithm or are about other algorithms
related to symbolic integration.  The plan is that someday, when I
finish implementing the algorithms from Bronstein's book
\cite{bronstein2005symbolic}, that I will work on these.
    \begin{itemize}
    \item \textbf{Bronstein} \cite{bronstein1989simplification}:  This
    is by the same author as and is referenced by Bronstein's book.  The
    paper contains an algorithm used in integrating tangents that was
    not included in the textbook.  Among the things that I am likely to
    implement in the future going beyond Bronstein's book, this is the
    most likely, as it is actually requirement for completing the
    algorithm (I believe Bronstein couldn't fit the material in his
    book, so he just referenced a paper that had written).

    \item \textbf{Davenport} \cite{davenport1984integration}: This
    source, like the next one, deals with the algebraic part of the
    Risch Algorithm.  The Risch Algorithm, being the complicated
    algorithm that it is, is actually broken up into three parts. 
    Bronstein's book only covers the first\footnote{First in the sense
    that the transcendental algorithm must be implemented before the
    algebraic algorithm can be.} and easiest part, the transcendental
    part.  The other two parts are the algebraic part, which Davenport
    details here, and the mixed transcendental-algebraic part (the
    transcendental part deals with purely transcendental functions, the
    algebraic part deals with purely algebraic functions, and the
    transcendental-algebraic part deals with functions that are
    combinations of both types).

    \item \textbf{Kauers} \cite{kauers2008integration}: This paper
    describes a simple heuristic algorithm for the part of the algebraic
    part.  The algebraic part of the algorithm is very difficult to
    implement---even more so than the transcendental part.  Therefore,
    sometimes implementors will use a heuristic instead of the full
    algorithm.  The downside to this is that a heuristic will be unable
    to prove when no antiderivative exists---it will only return with a
    failure.  This doesn't constitute a proof of nonexistence because it
    will sometimes also fail even when an antiderivative does exist,
    which is another downside of using a heuristic.  However, heuristics
    are easier to implement and often can act as fast preparsers to the
    full algorithm.
    

    \end{itemize}
\nocite{*}
\bibliographystyle{plain}
\bibliography{progress_report1}

\end{document}
