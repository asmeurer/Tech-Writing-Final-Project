\documentclass[12pt]{article}
\usepackage{amsmath}
\usepackage{amsfonts}
\usepackage[latin1]{inputenc}
\usepackage{amsmath}
\usepackage{amsfonts}
\usepackage{amssymb}
\usepackage{makeidx}
\usepackage{tabularx}
%\usepackage[T1]{fontenc}
\newcommand{\BibTeX}{{\sc Bib}\TeX}
\newcommand{\bibtex}{{\sc Bib}\TeX\ }
\newcommand{\latex}{\LaTeX\ }
\begin{document}
\title{Progress Report 1 for Report on the Risch Algorithm for Symbolic
Integration and Implementation in the Sym\-Py Computer Algebra System}
\author{Aaron Meurer}
\date{November 12, 2010}
\maketitle

% Include the following information:
% --Secondary research conducted
% --Primary research conducted (if any)
% --Any changes or modifications in your report topic or outline
% --Any changes to your time schedule
% --Any questions, concerns, etc.
% --A short, annotated bibliography with 5 secondary sources (in an
% appendix). For each source, provide a citation using whatever format
% you intend to use for the final report. Also, for each source, include
% a brief paragraph summarizing the source and explaining how you intend
% to use it in your report.

\section{Research Conducted}
\subsection{Primary Research}
For all intents and purposes, the primary research for my project has
already been conducted.  As I said in the proposal, I spent last summer
learning and implementing the Transcendental Risch Algorithm for
Symbolic Integration. Because I was implementing the algorithm, I had to
fully understand it, both from a mathematical and a implementational
perspective.  Therefore, I have read quite closely almost the entirety
of Bronstein's book \cite{bronstein2005symbolic}.  In other words, this was
more than just a cursory reading.  In order to implement an algorithm,
you have to understand exactly what is happening mathematically, so that
you can fix problems when they arrise.  Also, sometimes the algorithms
given were ambiguous, requiring me to understand what should be
happening enough to implement it, because actual implementations in
programming languages cannot be ambiguous in any way, by their nature.

\subsection{Secondary Research}
Secondary research is actually the thing that I need to do more of. I
still need to read through most of the citations that I go through in
Section \ref{annotated-bib}.  I think that some of these may end up
changing for the final report, depending on exactly what direction I
want to take with it.  For example, the sources I have now mostly relate
to things that I would want to do with the algorithm in the future that
go beyond Bronstein's book.  For now, that is the plan, because I don't
see many other things that I could do with secondary sources.  But, for
example, one thing that I might choose to do instead would be to look at
what other people in the field have done.  That would require a whole
different secondary source set.  This would be more difficult, though,
as it would require me reading source code for other programs, rather
than reading articles from journals.

\section{Modifications}

\section{Annotated Bibliography}
\label{annotated-bib}

\nocite{*}
\bibliographystyle{plain}
\bibliography{progress_report1}

\end{document}
