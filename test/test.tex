\documentclass[12pt,titlepage]{article}
\usepackage{amsmath}
\usepackage{amsfonts}
\usepackage[latin1]{inputenc}
\usepackage{amsmath}
\usepackage{amsfonts}
\usepackage{amssymb}
\usepackage{makeidx}
\usepackage{tabularx}
\usepackage{url}
\usepackage[toc,acronym,description]{glossaries}
\usepackage{cite}
\makeglossaries
\begin{document}
\newglossaryentry{something}{
name={something},
description={A glossary entry}
}

\newglossaryentry{something else}{
name={something else},
description={A glossary entry with a footnote\footnote{right here some math $e^{\frac{1}{2}\ln{x}}$ reference the glossary \gls{something else}}}
}

\newglossaryentry{transcendental}{
name={transcendental},
description={A function is \gls{transcendental} if it is not
\gls{algebraic}.  A function is \textit{purely \gls{transcendental}} if
it does not contain any \gls{algebraic} components.  An important result
from analysis is that the functions $e^x$, $\ln{x}$, $\sin{x}$,
$\cos{x}$, and $\tan{x}$ are all \gls{transcendental}.  Roughly
speaking, a function is \gls{transcendental} if it contains one of
these, and it is purely \gls{transcendental} if it does not contain any
radicals\footnote{There are exceptions to this rule.  For example,
$e^{\frac{1}{2}\ln{x}}$ is not transcendental because it is equivalent to
$\sqrt{x}$.  An important part of the \gls{transcendental} part of the
Risch Algorithm involves making sure that the integrand really is
\gls{transcendental}}.  
For example, $e^{x + 1}$ is purely
\gls{transcendental}, $\sqrt[3]{\ln{x}}$ is \gls{transcendental} but not
purely \gls{transcendental}, and $\sqrt{x}$ is neither
\gls{transcendental} nor purely \gls{transcendental} (it is
\gls{algebraic})}
}

\newglossaryentry{algebraic}{
name={algebraic},
description={A function is \gls{algebraic} if it is the root of a
polynomial with coefficients that are rational functions with rational
number coefficients.  For example, the function $\sqrt{x + 1}$ is
\gls{algebraic} because it is the root of the polynomial $y^2 = x + 1$. 
A function that is not \gls{algebraic} is called \gls{transcendental}}
}


\section{The text}
\Gls{something} and \gls{something else}.

\glsaddall
\printglossary

\end{document}
