\documentclass[12pt]{article}
\usepackage{amsmath}
\usepackage{amsfonts}
\usepackage[latin1]{inputenc}
\usepackage{amsmath}
\usepackage{amsfonts}
\usepackage{amssymb}
\usepackage{makeidx}
\usepackage{tabularx}
%\usepackage[T1]{fontenc}
\newcommand{\BibTeX}{{\sc Bib}\TeX}
\newcommand{\bibtex}{{\sc Bib}\TeX\ }
\newcommand{\latex}{\LaTeX\ }
\begin{document}
\title{Proposal (Draft)}
\author{Aaron Meurer}
\date{October 29, 2010}
\maketitle
\section{Summary}
For this project, I am going to study the Risch Algorithm for
integrating integration of transcendental equations.  The Risch
Algorithm is a complete algorithm for computing elementary
antiderivatives, or proving that no such antiderivative exists.  Over
the summer of 2010, I worked under the Goo\-gle Summer of Code program
implementing the transcendental part of the algorithm in Sym\-Py, a
computer algebra system (CAS) written in the Py\-thon programming
language.

Integration is a fundamental operation in mathematics.  Most sciences
that apply mathematics to themselves use calculus, which will invariably
involve integration and integrals.  The ability to algorithmically
compute symbolic integrals is therefore of extreme practical importance.
 Mathematically, it is also of great interest that there exists an
algorithm that not only can compute elementary symbolic integrals, but
also that can prove that no such one exists when that is the case.

\section{Outline}

\section{Outcome}

\section{Research Strategies}
\subsection{Primary Research}

\subsection{Secondary Research}
\label{secondary-research}

\section{Timeline}
Dates in bold are official due dates from the syllabus (revision as of October 28, 2010).  

\begin{itemize}
\item \textbf{October 29, 2010}: Draft of proposal (this document) due
in class.  Peer reviews in class.
\item Finish the proposal.
\item \textbf{November 1, 2010}: Proposal (this document) due in class. 
\item Start doing research.  Most of the research was already done last
summer, so this involve collecting the research together in a form
suitable for the report.
\item \textbf{November 8, 2010 - November 12, 2010}: Conference week. 
\item \textbf{???}: Progress report memo 1 due in class.  The
bibliography must contain at least five secondary sources by this point.
\footnote{This won't be a problem, as I already have five sources.  See
the section \ref{secondary-research} above and the References section
below.}
\item \textbf{November 19, 2010}: Progress report memo 2 due in class.
\item \textbf{November 24, 2010}: Draft of technical report due in
class.  Peer reviews in class.
\item \textit{November 25, 2010 - November 28, 2010}: Thanksgiving break.
\item \textbf{December 1, 2010}: Peer reviews in class.
\item \textbf{December 3, 2010}: Peer reviews in class.
\item \textbf{December 6, 2010}: Presentations in class.
\item \textbf{December 8, 2010}: Presentations in class.
\item \textbf{December 10, 2010}: Presentations in class.  Final
technical report and presentation materials due in class.
\end{itemize}

\section{Questions and Concerns}
\begin{enumerate}
\item When is the first progress report memo due?  The date is not given
in the most recent version of the syllabus online.
\end{enumerate}
\nocite{*}
\bibliographystyle{plain}
\bibliography{proposal}
\label{refs}
\end{document}
