\documentclass[12pt]{article}
\usepackage{amsmath}
\usepackage{amsfonts}
\usepackage[latin1]{inputenc}
\usepackage{amsmath}
\usepackage{amsfonts}
\usepackage{amssymb}
\usepackage{makeidx}
\usepackage{tabularx}
%\usepackage[T1]{fontenc}
\newcommand{\BibTeX}{{\sc Bib}\TeX}
\newcommand{\bibtex}{{\sc Bib}\TeX\ }
\newcommand{\latex}{\LaTeX\ }
\begin{document}
\title{Proposal (Draft)}
\author{Aaron Meurer}
\date{October 29, 2010}
\maketitle
\section{Summary}
For this project, I am going to study the Risch Algorithm for
integrating integration of transcendental equations.  The Risch
Algorithm is a complete algorithm for computing elementary
antiderivatives, or proving that no such antiderivative exists.  Over
the summer of 2010, I worked under the Goo\-gle Summer of Code program
implementing the transcendental part of the algorithm in Sym\-Py, a
computer algebra system (CAS) written in the Py\-thon programming
language.

Integration is a fundamental operation in mathematics.  Most sciences
that apply mathematics to themselves use calculus, which will invariably
involve integration and integrals.  The ability to algorithmically
compute symbolic integrals is therefore of extreme practical importance.
 Mathematically, it is also of great interest that there exists an
algorithm that not only can compute elementary symbolic integrals, but
also that can prove that no such one exists when that is the case.

\section{Outline}
Over the summer of 2010, I worked on implementing the
transcendental\footnote{Transcendental means that the functions handled
by this sub-part of the algorithm cannot contain algebraic functions. 
More or less, this means that the function cannot contain radical
expressions like $\sqrt{x + 1}$ or $\sqrt[3]{\ln{x}}$.  However, it can
contain exponentials ($e^x$), logarithms ($\ln{x}$), or trigonometric
functions ($\sin{x}$).} part of the Risch Integration Algorithm in
Sym\-Py, an open source computer algebra system written in Python.  The
algorithm is very complex and difficult to implement, and requires some
understanding of advanced mathematics to full understand.  This report
will contain an overview of my work, and will also look at some of the
other implementations of the algorithm in other open source computer
algebra systems.

Integration is one of the two important operators from calculus, the
other being differentiation.  Unlike differentiation, however, symbolic
integration (i.e., indefinite integration) of elementary functions is
not a straightforward process. The methods taught in calculus, such as
integration by substitution, integration by parts, trigonometric
integration, and trigonometric substitution are only heuristics that can
be applied to a special class of elementary functions.  

For example, a calculus student would have a hard time computing

\begin{equation}
\label{hard-integral}
\int{
    \frac{\left( 1 + e^{x^2} \right)
        \left(4x^3e^{x^2} - 4x^2e^{x^2}\ln{x} - x + 1 - xe^{x^2} +
        e^{x^2}\right)}
    {x\left(\ln{x} - x\right)^2}\,dx}
\end{equation}

even though the solution is the relatively simple 

\begin{equation}
\label{hard-integral-sol}
\frac{\left(1 + e^{x^{2}}\right)^{2}}{x - \ln{x}}.
\end{equation}

The problem is further complicated by the fact that, unlike the case
with differentiation, not all elementary function has an elementary
antiderivative\footnote{By the Fundamental Theorem of Calculus,
indefinite integration is the inverse of differentiation, hence, we also
sometimes call it antidifferentiation}.  For example, the function

\begin{equation}
\label{erf}
\int{e^{-x^2}dx}
\end{equation}

is not elementary, i.e., it can not be represented as a combination of
exponentials, logarithms, powers, and trig functions by addition,
subtraction, multiplication, division, composition.  Up to a constant
factor, equation \ref{erf} is known as the error function, and used
heavily in statistics\footnote{In particular, the error function
represents the cumulative distribution function of the normal
distribution (i.e., a bell curve), and its values are used to calculate
probabilities.  The fact that this function is non-elementary implies
that statistical computing packages must use numerical techniques to
calculate these values}.

It turns out that the Risch Algorithm not only gives a complete
algorithm for symbolic integration, even for integrals as complex as the
one given in equation \ref{hard-integral}, but it can also prove that no
elementary antiderivative can exist for the integral, as is the case
with equation \ref{erf}.

The algorithm I implemented in Sym\-Py can handle both of these case,
meaning that it can produce equation \ref{hard-integral-sol} given
equation \ref{hard-integral}, and it can prove that equation \ref{erf}
is non-elementary.

\section{Outcome}
The report will be typset using the \latex typesetting system, and the
bibliographies will be formatted automatically using \BibTeX.  The
format will be a standard report format.  It will have an abstract, an
introduction, and sections detailing the different parts of the report.
\section{Research Strategies}
\subsection{Primary Research}
My main source for the algorithm was the textbook by Manuel Bronstein
\cite{bronstein2005symbolic}.  I have read mos tof this book , and have
completed implementing most of the pseudocode algorithms given in it.

Because my report will also be on my implementation of the algorithm, I
will focus on my own source code from the summer.  I also plan on
looking at implementations of the same algorithm in other open source
CASs and comparing them to my own.
\subsection{Secondary Research}
\label{secondary-research}

\section{Timeline}
Dates in bold are official due dates from the syllabus (revision as of
October 28, 2010).

\begin{itemize}
\item \textbf{October 29, 2010}: Draft of proposal (this document) due
in class.  Peer reviews in class.
\item Finish the proposal.  Make changes based on peer review feedback.  
\item \textbf{November 1, 2010}: Proposal (this document) due in class. 
\item Start doing research.  Most of the research was already done last
summer, so this involve collecting the research together in a form
suitable for the report.
\item Start writing the report.  
\item \textbf{November 8, 2010 - November 12, 2010}: Conference week. 
\item At this point, I should have enough of the report done so that I
will have questions to bring to the conference.  Also, I should have a
very rough draft ready by this point to also bring to the conference.
\item \textbf{???}: Progress report memo 1 due in class.  The
bibliography must contain at least five secondary sources by this point
\footnote{This won't be a problem, as I already have five sources.  See
the section \ref{secondary-research} above and the References section
below.}.
\item Have something written for all the sections of the report.  Have
all secondary sources that will be used in the report.
\item \textbf{November 19, 2010}: Progress report memo 2 due in class.
\item Finish most major sections of the report.  Start working on the
presentation.
\item \textbf{November 24, 2010}: Draft of technical report due in
class.  Peer reviews in class.
\item Make changes from peer review feedback.
\item \textit{November 25, 2010 - November 28, 2010}: Thanksgiving break.
\item All but major parts of the report should be done by this point. 
Most parts of the presentation should be ready.
\item \textbf{December 1, 2010}: Peer reviews in class.
\item Make changes from peer review feedback.
\item \textbf{December 3, 2010}: Peer reviews in class.
\item Make changes from peer review feedback.
\item Have the presentation and the report finalized.
\item \textbf{December 6, 2010}: Presentations in class.
\item \textbf{December 8, 2010}: Presentations in class.
\item \textbf{December 10, 2010}: Presentations in class.  Final
technical report and presentation materials due in class.
\end{itemize}

\section{Questions and Concerns}
\begin{enumerate}
\item When is the first progress report memo due?  The date is not given
in the most recent version of the syllabus online.
\end{enumerate}
\nocite{*}
\bibliographystyle{plain}
\bibliography{proposal}
\label{refs}
\end{document}
