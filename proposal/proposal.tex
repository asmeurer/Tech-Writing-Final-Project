\documentclass[12pt]{article}
\usepackage{amsmath}
\usepackage{amsfonts}
\begin{document}
\title{Proposal (Draft)}
\author{Aaron Meurer}
\date{October 29, 2010}
\maketitle
\section{Summary}
For this project, I am going to study the Risch Algorithm for
integrating integration of transcendental equations.  The Risch
Algorithm is a complete algorithm for computing elementary
antiderivatives, or proving that no such antiderivative exists.  Over
the summer of 2010, I worked under the Google Summer of Code program
implementing the transcendental part of the algorithm in SymPy, a
computer algebra system (CAS) written in the Python programming
language.

Integration is a fundamental operation in mathematics.  Most sciences
that apply mathematics to themselves use calculus, which will invariably
involve integration and integrals.  The ability to algorithmically
compute symbolic integrals is therefore of extreme practical importance.
 Mathematically, it is also of great interest that there exists an
algorithm that not only can compute elementary symbolic integrals, but
also that can prove that no such one exists when that is the case.

\section{Outline}

\section{Outcome}

\section{Research Strategies}
\subsection{Primary Research}

\subsection{Secondary Research}

\section{Timeline}

\section{Questions and Concerns}


\end{document}
