\documentclass{beamer}

\usetheme{Torino}
\AtBeginSection[] 
{ 
  \begin{frame}<beamer> 
    \frametitle{Agenda} 
    \tableofcontents[currentsection] 
  \end{frame} 
}  % for recurrent Agenda slide
\numberwithin{equation}{section} % Number equations with sections



\usepackage{amsmath}
\usepackage{amsfonts}
\usepackage[latin1]{inputenc}
\usepackage{amsmath}
\usepackage{amsfonts}
\usepackage{amssymb}
\usepackage{makeidx}
\usepackage{tabularx}
\usepackage{url}
\usepackage[toc,acronym,description]{glossaries}
\usepackage{cite}
\usepackage{hyperref}
\usepackage[noend,boxed,fillcomment]{algorithm2e}
\usepackage{multicol}
\usepackage{fancyvrb}
\usepackage{color}
\usepackage{subfigure}
% use \usepackage[pdfborder=0in]{hyperref} instead to disable red box links
%\usepackage[T1]{fontenc}
\newcommand{\BibTeX}{{\sc Bib}\TeX}
\newcommand{\rischintegrate}{\texttt{risch\_integrate()}}
\hyphenation{Sym-Py an-ti-der-iv-a-tive an-ti-der-iv-a-tives
an-ti-diff-er-en-tia-tion Goo-gle arc-trig-o-no-met-ric
non-el-e-men-tary}

\title{Report on the Risch Algorithm for Symbolic
Integration and Implementation in the SymPy Computer Algebra System}
\author{Aaron Meurer}
\date{December 9, 2010}

\begin{document}

\begin{frame}
    \titlepage
\end{frame}

\begin{frame} 
    \frametitle{Agenda} 
    \tableofcontents 
\end{frame} 

\section{The Risch Algorithm}

\subsection{Liouville's Theorem}

\begin{frame}
    \frametitle{Liouville's Theorem}
    \begin{itemize}
    \item something
    \end{itemize}
\end{frame}

\subsection{The Transcendental Algorithm}

\begin{frame}
    something
\end{frame}

\section{Implementation in SymPy}

\begin{frame}
    \begin{figure}
    \subfigure{\includegraphics[width=.3\textwidth]{./python-logo-master-v3-TM.png}}
    \subfigure{\includegraphics[width=.3\textwidth]{./sympy-160px.png}}
   \subfigure{\includegraphics[width=.3\textwidth]{./GSoC_2010_logo/2010_NoURL_950x846px.png}}
   \end{figure}

    Over the summer of 2010, I worked for the Python Software
    Organization with the SymPy project under the Google Summer of Code
    program.
\end{frame}

\section{Future}

\begin{frame}

\end{frame}


\section{Questions}

\begin{frame}
    \frametitle{Questions?}
    \huge{Questions?}
\end{frame}

\end{document}
